%%%%%%%%%%%%%%%%%%%%%%%%%%%%%%%%%%%%%%%%%
% Wenneker Article
% LaTeX Template
% Version 2.0 (28/2/17)
%
% This template was downloaded from:
% http://www.LaTeXTemplates.com
%
% Authors:
% Vel (vel@LaTeXTemplates.com)
% Frits Wenneker
%
% License:
% CC BY-NC-SA 3.0 (http://creativecommons.org/licenses/by-nc-sa/3.0/)
%
%%%%%%%%%%%%%%%%%%%%%%%%%%%%%%%%%%%%%%%%%

%----------------------------------------------------------------------------------------
%	PACKAGES AND OTHER DOCUMENT CONFIGURATIONS
%----------------------------------------------------------------------------------------

\documentclass[10pt, a4paper, twocolumn]{article}

\input{structure.tex} % Specifies the document structure and loads requires packages

%----------------------------------------------------------------------------------------
%	ARTICLE INFORMATION
%----------------------------------------------------------------------------------------

\title{My favorite physicist} % The article title

\author{
	\authorstyle{Group for Lasers and Plasmas\textsuperscript{$\ast,\dagger,\ddagger$}} % Authors
	\newline\newline % Space before institutions
	\textsuperscript{$\ast$}\institution{Institute for Plasmas and Nuclear Fusion}\\
	\textsuperscript{$\dagger$}\institution{Instituto Superior Técnico, Lisboa}\\
	\textsuperscript{$\ddagger$}\institution{\url{http://epp.ist.utl.pt/}}
}


\date{May 5, 2017}
%----------------------------------------------------------------------------------------


\begin{document}

\maketitle % Print the title

\thispagestyle{firstpage} % Apply the page style for the first page (no headers and footers)

%----------------------------------------------------------------------------------------
%	ABSTRACT
%----------------------------------------------------------------------------------------

\lettrineabstract{This is a small example of a showcase for a session on git/github. The main idea is to use some resources available only and add a section of one famous physicist. You can even use pictures of equations or something you associate with this person. It does not matter for now. Just use some reference if you are copying something. Everyone likes to be acknowledged.}

%----------------------------------------------------------------------------------------
%	ARTICLE CONTENTS
%----------------------------------------------------------------------------------------
\section*{Instructions for authors}

A description on how to work with the repository can be found on \href{https://github.com/ahelm/my-favorite-physicist}{github}. Below you can find an example how to add one physicist you like to this document. You can use images or equations inside the document, but if you take some text from an online resource be so kind and provide where you have taken it from.

The directory structure is designed to reduce conflicts and create a better overview. The directory \texttt{/content} includes the different \texttt{.tex} files and you should add the corresponding \LaTeX\ files over there. Please store the images in the folder \texttt{content/figures}.

\input{content/albert_einstein.tex}

\input{content/feynman.tex}

\end{document}
