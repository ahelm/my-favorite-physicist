%%%%%%%%%%%%%%%%%%%%%%%%%%%%%%%%%%%%%%%%%
% Wenneker Article
% LaTeX Template
% Version 2.0 (28/2/17)
%
% This template was downloaded from:
% http://www.LaTeXTemplates.com
%
% Authors:
% Vel (vel@LaTeXTemplates.com)
% Frits Wenneker
%
% License:
% CC BY-NC-SA 3.0 (http://creativecommons.org/licenses/by-nc-sa/3.0/)
%
%%%%%%%%%%%%%%%%%%%%%%%%%%%%%%%%%%%%%%%%%

%----------------------------------------------------------------------------------------
%	PACKAGES AND OTHER DOCUMENT CONFIGURATIONS
%----------------------------------------------------------------------------------------

\documentclass[10pt, a4paper, twocolumn]{article}

%%%%%%%%%%%%%%%%%%%%%%%%%%%%%%%%%%%%%%%%%
% Wenneker Article
% Structure Specification File
% Version 1.0 (28/2/17)
%
% This file originates from:
% http://www.LaTeXTemplates.com
%
% Authors:
% Frits Wenneker
% Vel (vel@LaTeXTemplates.com)
%
% License:
% CC BY-NC-SA 3.0 (http://creativecommons.org/licenses/by-nc-sa/3.0/)
%
%%%%%%%%%%%%%%%%%%%%%%%%%%%%%%%%%%%%%%%%%
%we were here!! :D FERMI 

%----------------------------------------------------------------------------------------
%	PACKAGES AND OTHER DOCUMENT CONFIGURATIONS
%----------------------------------------------------------------------------------------

\usepackage[english]{babel} % English language hyphenation

\usepackage{microtype} % Better typography

\usepackage{amsmath,amsfonts,amsthm} % Math packages for equations

\usepackage[svgnames]{xcolor} % Enabling colors by their 'svgnames'

\usepackage[hang, small, labelfont=bf, up, textfont=it, position=below, skip=5pt]{caption} % Custom captions under/above tables and figures
\setlength{\belowcaptionskip}{-10pt}

\usepackage{booktabs} % Horizontal rules in tables

\usepackage{lastpage} % Used to determine the number of pages in the document (for "Page X of Total")

\usepackage{graphicx} % Required for adding images

\usepackage{enumitem} % Required for customising lists
\setlist{noitemsep} % Remove spacing between bullet/numbered list elements

\usepackage{sectsty} % Enables custom section titles
\allsectionsfont{\usefont{OT1}{phv}{b}{n}} % Change the font of all section commands (Helvetica)

\usepackage[pdfborder={0 0 0},colorlinks=true,linkcolor={black},citecolor={black},urlcolor={blue!60!black}]{hyperref}

%----------------------------------------------------------------------------------------
%	MARGINS AND SPACING
%----------------------------------------------------------------------------------------

\usepackage{geometry} % Required for adjusting page dimensions

\geometry{
	top=1cm, % Top margin
	bottom=1.5cm, % Bottom margin
	left=2cm, % Left margin
	right=2cm, % Right margin
	includehead, % Include space for a header
	includefoot, % Include space for a footer
	%showframe, % Uncomment to show how the type block is set on the page
}

\setlength{\columnsep}{7mm} % Column separation width

%----------------------------------------------------------------------------------------
%	FONTS
%----------------------------------------------------------------------------------------

\usepackage[T1]{fontenc} % Output font encoding for international characters
\usepackage[utf8]{inputenc} % Required for inputting international characters

\usepackage{XCharter} % Use the XCharter font

%----------------------------------------------------------------------------------------
%	HEADERS AND FOOTERS
%----------------------------------------------------------------------------------------

\usepackage{fancyhdr} % Needed to define custom headers/footers
\pagestyle{fancy} % Enables the custom headers/footers

\renewcommand{\headrulewidth}{0.0pt} % No header rule
\renewcommand{\footrulewidth}{0.4pt} % Thin footer rule

\renewcommand{\sectionmark}[1]{\markboth{#1}{}} % Removes the section number from the header when \leftmark is used

%\nouppercase\leftmark % Add this to one of the lines below if you want a section title in the header/footer

% Headers
\lhead{} % Left header
\chead{\textit{\thetitle}} % Center header - currently printing the article title
\rhead{} % Right header

% Footers
\lfoot{} % Left footer
\cfoot{} % Center footer
\rfoot{\footnotesize Page \thepage\ of \pageref{LastPage}} % Right footer, "Page 1 of 2"

\fancypagestyle{firstpage}{ % Page style for the first page with the title
	\fancyhf{}
	\renewcommand{\footrulewidth}{0pt} % Suppress footer rule
}

%----------------------------------------------------------------------------------------
%	TITLE SECTION
%----------------------------------------------------------------------------------------

\newcommand{\authorstyle}[1]{{\large\usefont{OT1}{phv}{b}{n}\color{DarkRed}#1}} % Authors style (Helvetica)

\newcommand{\institution}[1]{{\footnotesize\usefont{OT1}{phv}{m}{sl}\color{Black}#1}} % Institutions style (Helvetica)

\usepackage{titling} % Allows custom title configuration

\newcommand{\HorRule}{\color{DarkGoldenrod}\rule{\linewidth}{1pt}} % Defines the gold horizontal rule around the title

\pretitle{
	\vspace{-30pt} % Move the entire title section up
	\HorRule\vspace{10pt} % Horizontal rule before the title
	\fontsize{32}{36}\usefont{OT1}{phv}{b}{n}\selectfont % Helvetica
	\color{DarkRed} % Text colour for the title and author(s)
}

\posttitle{\par\vskip 15pt} % Whitespace under the title

\preauthor{} % Anything that will appear before \author is printed

\postauthor{ % Anything that will appear after \author is printed
	\vspace{10pt} % Space before the rule
	\par\HorRule % Horizontal rule after the title
	\vspace{20pt} % Space after the title section
}

%----------------------------------------------------------------------------------------
%	ABSTRACT
%----------------------------------------------------------------------------------------

\usepackage{lettrine} % Package to accentuate the first letter of the text (lettrine)
\usepackage{fix-cm}	% Fixes the height of the lettrine

\newcommand{\initial}[1]{ % Defines the command and style for the lettrine
	\lettrine[lines=3,findent=4pt,nindent=0pt]{% Lettrine takes up 3 lines, the text to the right of it is indented 4pt and further indenting of lines 2+ is stopped
		\color{DarkGoldenrod}% Lettrine colour
		{#1}% The letter
	}{}%
}

\usepackage{xstring} % Required for string manipulation

\newcommand{\lettrineabstract}[1]{
	\StrLeft{#1}{1}[\firstletter] % Capture the first letter of the abstract for the lettrine
	\initial{\firstletter}\textbf{\StrGobbleLeft{#1}{1}} % Print the abstract with the first letter as a lettrine and the rest in bold
}
 % Specifies the document structure and loads requires packages

%----------------------------------------------------------------------------------------
%	ARTICLE INFORMATION
%----------------------------------------------------------------------------------------

\title{My favorite physicist} % The article title

\author{
	\authorstyle{Group for Lasers and Plasmas\textsuperscript{$\ast,\dagger,\ddagger$}} % Authors
	\newline\newline % Space before institutions
	\textsuperscript{$\ast$}\institution{Institute for Plasmas and Nuclear Fusion}\\
	\textsuperscript{$\dagger$}\institution{Instituto Superior Técnico, Lisboa}\\
	\textsuperscript{$\ddagger$}\institution{\url{http://epp.ist.utl.pt/}}
}


\date{May 5, 2017}
%----------------------------------------------------------------------------------------


\begin{document}

\maketitle % Print the title

\thispagestyle{firstpage} % Apply the page style for the first page (no headers and footers)

%----------------------------------------------------------------------------------------
%	ABSTRACT
%----------------------------------------------------------------------------------------

\lettrineabstract{This is a small example of a showcase for a session on git/github. The main idea is to use some resources available only and add a section of one famous physicist. You can even use pictures of equations or something you associate with this person. It does not matter for now. Just use some reference if you are copying something. Everyone likes to be acknowledged.}

%----------------------------------------------------------------------------------------
%	ARTICLE CONTENTS
%----------------------------------------------------------------------------------------
\input{content/instruction.tex}

\input{content/albert_einstein.tex}
\input{content/fermi.tex}

\section*{Richard Feynman\protect\footnote{\url{https://en.wikipedia.org/wiki/Richard_Feynman}}}

%\begin{figure}[ht]
%  \centering
%  \includegraphics[width=0.8\linewidth]{content/figures/albert_einstein_picuture.jpg}
%  \caption{Albert Einstein in 1921\protect\footnotemark}
%\end{figure}
%\footnotetext{\url{https://commons.wikimedia.org/wiki/File:Einstein_1921_by_F_Schmutzer_-_restoration.jpg}}

Richard Phillips Feynman (May 11, 1918 – February 15, 1988) was an American theoretical physicist known for his work in the path integral formulation of quantum mechanics, the theory of quantum electrodynamics, and the physics of the superfluidity of supercooled liquid helium, as well as in particle physics for which he proposed the parton model. For his contributions to the development of quantum electrodynamics, Feynman, jointly with Julian Schwinger and Shin'ichiro Tomonaga, received the Nobel Prize in Physics in 1965.

Feynman developed a widely used pictorial representation scheme for the mathematical expressions governing the behavior of subatomic particles, which later became known as Feynman diagrams. During his lifetime, Feynman became one of the best-known scientists in the world. In a 1999 poll of 130 leading physicists worldwide by the British journal Physics World he was ranked as one of the ten greatest physicists of all time.[1]

He assisted in the development of the atomic bomb during World War II and became known to a wide public in the 1980s as a member of the Rogers Commission, the panel that investigated the Space Shuttle Challenger disaster. Along with his work in theoretical physics, Feynman has been credited with pioneering the field of quantum computing and introducing the concept of nanotechnology. He held the Richard C. Tolman professorship in theoretical physics at the California Institute of Technology.

Feynman was a keen popularizer of physics through both books and lectures, including a 1959 talk on top-down nanotechnology called There's Plenty of Room at the Bottom, and the three-volume publication of his undergraduate lectures, The Feynman Lectures on Physics. Feynman also became known through his semi-autobiographical books Surely You're Joking, Mr. Feynman! and What Do You Care What Other People Think? and books written about him, such as Tuva or Bust! and Genius: The Life and Science of Richard Feynman by James Gleick.


\end{document}
